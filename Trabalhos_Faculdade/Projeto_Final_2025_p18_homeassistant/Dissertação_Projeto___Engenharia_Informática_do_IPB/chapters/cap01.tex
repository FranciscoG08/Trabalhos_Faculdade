\chapter{Introdução}\label{cap:intro}


A automação residencial tem-se destacado como uma área de grande potencial de crescimento, proporcionando aos moradores um maior controlo sobre as suas casas através de dispositivos conectados, oferecendo benefícios em termos de conveniência e gestão eficiente de recursos. Este projeto surge como uma contribuição para a evolução dessa área, ao desenvolver um sistema integrado de automação residencial que centraliza o controlo de diversos dispositivos inteligentes numa única plataforma.

\section{Enquadramento}

É inegável que, nos últimos anos, o mercado dos dispositivos inteligentes tem crescido a um ritmo acelerado. O aumento das soluções de domótica, impulsionado pela popularização das casas inteligentes e do interesse em controlar as habitações são das principais razões. Estes dispositivos inteligentes, cada vez mais, fazem parte do nosso quotidiano, melhorando a maneira como interagimos com a casa, proporcionando um maior conforto aos utilizadores, assim como uma gestão dos gastos energéticos e na segurança da habitação.
A instabilidade nos preços da energia e uma maior preocupação ambiental têm levado os consumidores a procurar soluções \gls{IoT} que permitam uma utilização mais racional dos recursos.

Apesar deste crescimento dos dispositivos \gls{IoT} e do elevado interesse das pessoas, muitos utilizadores ainda enfrentam dificuldades na integração e configuração com a plataforma escolhida. Muitas vezes existe falta de compatibilidade entre a plataforma escolhida e os dispositivos \gls{IoT}.

Para resolver estes problemas, surgiu o \gls{HA}, uma solução open-source que conta com uma vasta lista de dispositivos compatíveis e uma grande comunidade que desenvolve e mantém as integrações. Um dos pontos fortes desta plataforma é o facto de não depender de serviços externos, o que garante total privacidade e controlo dos dados.

Como já referido, uma das grandes vantagens do \gls{HA} é a sua comunidade ativa, sempre atualizando e disponibilizando novas e melhoradas soluções. Esta comunidade é fundamental para manutenção e evolução do \gls{HA}. Com o empenho desta comunidade, o \gls{HA} tornou-se uma das melhores opções para soluções de domótica, sendo extremamente estável, versátil e compatível com a maioria dos dispositivos existentes no mercado. Tudo isto tem atraído cada vez mais utilizadores e desenvolvedores à plataforma.


\section{Objetivos}

O foco deste projeto é demonstrar de uma maneira simples, a capacidade do \gls{HA} para criar um sistema de domótica simples e centralizado. Para isso criou-se um sistema que permite uma observação gráfica do consumo/produção de energia da casa, visualização das câmaras de segurança e um controlo das temperaturas através de termostatos. Desta forma, podemos garantir um maior conhecimento dos gastos de energia e um controlo da habitação.


\section{Estrutura do Documento}

O relatório está organizado da seguinte forma:
\begin{itemize}
  \item \textbf{Capítulo 1: Introdução} \\
  Apresenta o contexto do projeto, os seus objetivos e a estrutura geral do documento.

  \item \textbf{Capítulo 2: Contexto e Tecnologias/Ferramentas} \\
  Descreve as tecnologias utilizadas no desenvolvimento do sistema, com destaque para a plataforma \gls{HA} e a linguagem \gls{YAML}, explicando o seu papel na integração e personalização das automações.

  \item \textbf{Capítulo 3: Análise de Requisitos} \\
  Detalha o processo de implementação, desde a integração dos dispositivos inteligentes até à configuração das automações e personalização da interface gráfica.

  \item \textbf{Capítulo 4: Desenvolvimento/Implementação} \\
  Apresenta os resultados obtidos, acompanhados de uma análise crítica sobre o desempenho, usabilidade e eficácia da solução desenvolvida.

  \item \textbf{Capítulo 5: Conclusões} \\
  Resume as principais conclusões do projeto e sugere possíveis melhorias ou desenvolvimentos futuros com base na experiência adquirida.
\end{itemize}