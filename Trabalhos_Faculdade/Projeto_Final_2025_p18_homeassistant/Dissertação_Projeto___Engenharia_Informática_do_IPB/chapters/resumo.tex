%\thispagestyle{empty}

Vivemos numa época em que a sociedade está mais dependente da tecnologia, utilizando-a, por exemplo, para automatizar, controlar e gererir ambientes residenciais. A automação residencial, uma das áreas em maior crescimento a nível mundial, tem conquistado espaço ao oferecer soluções inovadoras que promovem conforto, segurança e eficiência nas habitações.

Este projeto visa o desenvolvimento de um sistema de automação residencial para integrar e centralizar diversos dispositivos e sistemas inteligentes, com o objetivo de melhorar o conforto, a segurança e a eficiência da habitação. A comunicação entre os dispositivos inteligentes é realizada exclusivamente através de conectividade Wi-Fi, garantindo uma configuração simples e sem fios.

Para isso, foi utilizada a plataforma \gls{HA}, que foi escolhida pela sua flexibilidade e capacidade de integrar uma ampla gama de dispositivos inteligentes, permitindo o controlo centralizado de diferentes sistemas. Além disso, foi utilizado o \gls{YAML} para a configuração e personalização do sistema, proporcionando uma maneira simples e legível de definir as integrações e automações. A combinação destas ferramentas permitiu o desenvolvimento de uma solução eficaz para o controlo inteligente de ambientes residenciais.


\mbox{}\linebreak
\noindent {\bf Palavras-chave:} home assistant, yaml, automação, ambientes inteligentes.


%\vfill
%\pagebreak
%\mbox{}
%\vfill
%\pagebreak